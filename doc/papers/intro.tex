%
% An Introduction to the Beetle Forth Virtual Processor
%
% Short introductory paper
%
% Reuben Thomas   13-24/11/96
%


\documentclass{article}
\usepackage[english]{babel}
\usepackage{a4,newlfont}


% Alter some default parameters for general typesetting

\frenchspacing


% New commands

\newcommand{\conc}[1]{\texttt{\textup{#1}}}



\title{An Introduction to the Beetle Forth Virtual Processor}
\author{Reuben Thomas}
\date{24th November 1996}
\begin{document}
\maketitle

\begin{abstract}
Beetle is a virtual processor designed for the Forth language. It uses a
modified byte-stream code designed for efficient execution which is binary
portable between implementations. It has been implemented in C and assembler.
The C implementation is completely machine-independent with the exception of
interactive input and output; the assembler version runs the supplied Forth
compiler at up to half the speed of the corresponding native code compiler
and generates more compact code. Beetle is designed to be embedded in other
programs; a simple debugger has been written to demonstrate this ability.
Beetle can be configured to perform bounds checking on all memory references.
A standard I/O library is implemented; access to native code routines is also
possible, allowing Forth and C programs to call each other.
\end{abstract}



\section{Introduction}

Ever since the invention of high-level languages, one of the most popular
implementation methods has been to use a virtual processor, often called an
interpreter~\cite{intcoproc}. A virtual processor, or ``VP'', usually
resembles a real processor, reading a virtual instruction stream and calling
appropriate routines to carry out the actions of the virtual instructions.
VPs are often designed to support a particular language, so that once a
VP has been produced writing a compiler is straightforward, and porting the
compiler requires little more effort than porting the VP. This method is
commonly used for functional languages such as LISP and
Miranda~\cite{lispemul,combreduc}, and especially for prototyping new
compilers~\cite{parscheme,aplgen}.

Beetle is just such a VP, and the language it supports is
Forth~\cite{starting4th}. Forth is a simple stack-based language which is
most commonly used in embedded systems; however, Beetle is more suited to
general applications, Forth compiler development and teaching, as it lacks
the speed and low-level hardware access required for embedded systems.

Add to the discussion above the fact that Forth is often implemented on top
of a virtual machine~\cite{threaded}, and it is hard to see what could be
interesting or novel about Beetle. The rest of this paper is an attempt to
show that Beetle is worthy of interest, but before proceeding to describe it,
three points should be made. First, while none of Beetle's features is
revolutionary, they have not to the author's knowledge been combined
before.\footnote{The same is true of the much more illustrious Java virtual
processor; indeed, the fact that the techniques involved in its construction
are all tried and tested is its greatest strength.} Secondly, virtual
processors are often forgotten as they lie buried under the compilers they
were designed to support, and there are few detailed reports available on the
design and performance of specific VPs (\cite{intcoproc} contains a good
bibliography of those which do exist).\footnote{Recent literature includes
interesting reports on new types of virtual processor~\cite{virttime}, but
while these are innovative and worthy of note, they are mainly relevant to
researchers in their application areas, and not to VP designers in general.}
Thirdly, after a period of quiescence, VPs are undergoing something of a
renaissance, and must be reassessed in the context of the new uses to which
they are put and machines on which they are run.



\section{The virtual processor design}

Beetle's architecture is similar to that of a real processor. It is
stack-based, and all computation takes place on the data stack, so it has no
general-purpose registers. There is also a return stack. All instructions are
represented by one-byte opcodes. Instructions take either zero or one
operand.


\subsection{Memory}

The memory is an array of four-byte words. The bytes in a word may be stored
in either little-endian or big-endian order, so word addressing is
always efficient, but byte addressing is little-endian, so that it is
identical in all implementations. The penalty is a single machine instruction
on big-endian machines to invert the bottom two bits of the address of a byte
reference.


\subsection{Registers}

Beetle has a program counter, \conc{EP} (``\conc{E}xecution \conc{P}ointer'')
and two stack pointers: \conc{SP}, the data \conc{S}tack \conc{P}ointer, and
\conc{RP}, the \conc{R}eturn stack \conc{P}ointer. \conc{I} holds the current
\conc{I}nstruction, and \conc{A}, the instruction \conc{A}ccumulator, the
next few instructions to execute. There are several other more specialised
registers.


\subsection{Execution}

When Beetle is started, it performs the following execution cycle:

\begin{it}
\begin{tabbing}
\hspace{0.5in}begin\\
\hspace{0.75in}copy the least-significant byte of \conc{A} to \conc{I}\\
\hspace{0.75in}shift \conc{A} arithmetically 8 bits to the right\\
\hspace{0.75in}execute the instruction in \conc{I}\\
\hspace{0.5in}repeat
\end{tabbing}
\end{it}

\noindent This demonstrates Beetle's main adaptation for efficient execution
on modern pro\-cessors, which is to load instructions severally rather than
singly. \conc{A} is four bytes wide; on most modern processors it is at least
as fast to load a four-byte word from memory as to load four single bytes.
When the accumulator becomes empty the value zero or 255 is copied to
\conc{I}; these are the opcodes of the instruction \conc{NEXT}, which causes
the word pointed to by \conc{EP} to be loaded into \conc{A}.

It might seem that \conc{NEXT} would therefore be executed at least every
fifth cycle, but since instructions such as branches perform an implicit
instruction fetch it is in fact only executed as about 10\% of instructions
obeyed.


\subsection{Operands}

The only operands are numeric literals and branch addresses. Where possible
these are packed into the instruction word directly after the instruction
opcode; addresses are turned into offsets for compactness. Such immediate
operands always occupy the rest of the instruction word. If the operand is
too big, then it is placed in the next available word; further instruction
opcodes may still be placed in the current word.

In the execution cycle \conc{A} is shifted arithmetically rather than
logically. This allows negative literals and branch offsets to be used
without needing extra opcodes. By shifting \conc{A} before the instruction is
executed immediate operands are accessible to their instructions with no
further decoding required.


\subsection{Implementability}

Because of its simple design which uses quantities no smaller than a byte
and no bigger than a four-byte word, Beetle is easy to implement, whether in
a high-level language or assembler. In a high-level implementation, Beetle's
registers map obviously on to variables and the memory can be represented as
a byte array, manipulated with array operations. All these operations are
optimised well by optimising compilers. In assembler Beetle's registers map
naturally on to machine registers, and ordinary memory addressing
instructions can be used to manipulate its memory.

The non-recursive design means that static allocation techniques can be used,
which keeps assembler implementations simple and makes them more likely to be
correct. The use of twos-complement arithmetic again leads to a natural and
efficient implementation; few computers still use other forms of arithmetic.



\section{Portability}

The greatest benefit of Beetle's implementability in high-level languages is
that it can easily be made portable. This was the main goal of the C
implementation, which is written entirely in ANSI C, and uses the standard
libraries almost exclusively. The only exception was forced by the nature of
Forth: since it is interactive, it requires unbuffered character input and
output. These can sometimes be achieved using the ANSI libraries, but it is
not guaranteed; on most operating systems (apart from UNIX) simple routines
are available to input and output single characters.

Thus character input and output macros must be supplied along with the
endianness and a few other machine characteristics for each machine on which
Beetle is to be compiled. The configuration is isolated in a special header
file, several of which have been prepared for various operating systems.
Special arrangements are made for UNIX, and extra code is supplied to
implement unbuffered input and output.

C Beetle has to date been compiled and tested successfully on five different
operating systems, including three versions of UNIX.

Beetle's simple design also makes it easy to write hand-coded implementations
quickly: the ARM version was completed in a week. This is still a short time
to transport an entire Forth system to a new environment, and as will be seen
in section~\ref{performance}, it gives markedly better performance.


\section{Compiler support}
\label{support}

Conventional Forth compilers use indirect threaded code, in which Forth words
(the equivalent of functions or procedures in other languages) are compiled
as lists of addresses pointing to other words. Each word has a code field,
which contains a pointer to code to execute the word; in most words, this is
the address interpreter, a minimal VP which interprets the list of addresses,
finding their code fields in turn and branching to them. In primitive words
written in assembler the code field will point to the corresponding machine
code.

This VP is so simple that it consists of just a few instructions. It is
flexible in that it can be adapted to many designs of compiler (and indeed,
many languages). On the other hand its code is not portable, nor particularly
dense, as even primitive instructions such as those provided in Beetle's
instruction set occupy a machine word, generally four bytes on modern
processors.\footnote{It is often claimed that Forth translates into
particularly dense object code, but this is doubtful on modern machines. One
explanation given is that Forth programmers simply write smaller programs
than programmers using other languages; it should also be borne in mind that
Forth has a much bigger advantage over other languages on 8-bit processors,
on which addresses are only sixteen bits long, and relatively much denser
compared with machine code than on 32-bit or 64-bit processors.}

By providing a range of specialised instructions (most of them corresponding
directly to ANSI Standard words), Beetle allows compact and portable code to
be generated. It also simplifies the implementation of compilers, at the
expense of fixing some design choices such as the number of stacks, and
forcing the use of the return stack for loop indices and counts.

Beetle provides most of the ANSI Core Word Set arithmetic, logical and memory
access words as native instructions, and directly supports all the usual
Forth looping constructs, as well as the \conc{CREATE\dots DOES>} datatype
declaration mechanism. Support is also provided for exceptions, though not
for local variables. This practice of providing instructions with relatively
high semantic content lessens the overheads of interpretation, as a lower
proportion of time is spent fetching and decoding instructions.



\section{Embedding and safety}

Beetle is designed to be used as an interpreter embedded in other programs.
The C implementation provides a header that programs may include to use
Beetle; at the moment only one instantiation of the interpreter is allowed,
but this restriction could easily be lifted. The memory and all the registers
are available to the program to be inspected and manipulated, so the C
program can both control Beetle, and, by means of Beetle's \conc{LINK}
instruction, be called by it, arguments and return values being passed on
Beetle's data stack. The facilities provided are enough to write a debugger,
and a simple debugger was written to aid the development of the C
implementation of Beetle and the porting of the Forth compiler to Beetle.

One of Beetle's registers, \conc{CHECKED}, controls whether address checking
is performed; in the C and assembler implementations its value is fixed at
compile time. When enabled, address checking is performed on all memory
references, and out-of-bounds and unaligned memory references are trapped and
reported to the calling program. In this situation, Beetle cannot corrupt the
program that controls it directly, though it can cause corruption or a crash
by indiscriminate use of \conc{LINK}.

These features make Beetle a good candidate for an embedded interpreter,
whether to implement an application-specific scripting language, or to
provide a safe way for an application to generate and run code on the fly.



\section{Performance}
\label{performance}

Beetle cannot hope to outperform native machine code because of the
interpretive overhead. The most important question to address is whether it
runs fast enough; it is also interesting to see what the interpretive
slow-down is.

Three main benchmarks were run: two computation-intensive prime-finding
programs, and a compiler testing program, which was read from disk as it
progressed, and also exercised the input-output functions quite heavily. The
timings discussed below are those taken when Beetle was peforming address
checks.

The timings were taken on a machine with an ARM610 processor rated at about
20mips. The C implementation of Beetle ranged between about 0.4mips and
0.5mips, and the native implementation averaged around 1mips; the raw
interpretive slow-down is between 20 and 40 times. The actual slow-down was
much less: when the benchmarks were run on the native ARM version of the
Forth compiler, they ran only 2.3 times faster than native Beetle, and 7.9
times faster than C Beetle. It should also be remembered that Beetle
input-output operations, such as reading bytes from a file, count as a single
interpretive instruction.

The tests also completed in a reasonable time on the native Beetle, though on
C Beetle they were rather slow, taking 200s to find all the primes up to
800,000, compared with 54s for native Beetle, 20s for a native Forth compiler
and 2.0s for a C translation, compiled with optimisations. Only Beetle
performed address checks, and the Forth compilers did not optimise; indeed,
Beetle's virtual code cannot be optimised much. This is an advantage inasmuch
as a na\"{\i}ve compiler will produce near-optimal code; however, since the
instructions are low-level, many are executed (67.8 million in the benchmark
under discussion) and the interpretive overhead remains high. This contrasts
with languages such as APL, in which VPs compete with native compilers for
speed since they spend far more time performing instructions than decoding
them~\cite{intcoproc}.

Nevertheless, Beetle is more than adequately fast for interactive program
development in the usual Forth style; since its compiler is ANSI-compliant,
demanding programs developed on it could be recompiled with an optimising
compiler to obtain better performance.



\section{Conclusion}

The Beetle system allows Forth programs to be developed in a safe
environment. With the addition of some more input-output primitives, and
extra instructions to support floating point arithmetic and the few other
unsupported parts of the ANSI Forth Standard, the compiler could easily be
extended to provide a full Forth development environment. Programs are
instantly portable to a wide range of machines, and ports to new machines are
simple. With a little more work, execution can be dramatically improved by
hand-coding the VP. Interworking with C and other high-level languages is
also straightforward, and Beetle can be embedded in application programs to
drive a command language or run code generated on the fly.

Beetle's design is simple, but closely adapted to modern processors. It is
easy to implement in high-level languages or assembler, and performs well in
both cases. At the beginning of a new era of VPs, Beetle is a useful starting
point.



\bibliographystyle{plain}
\bibliography{vm}


\end{document}
