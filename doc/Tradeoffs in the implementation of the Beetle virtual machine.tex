%
% Tradeoffs in the implementation of the Beetle virtual machine
%
% Reuben Thomas   28/5-1/7/96
% copyright banner added 31/1/98
% made HeVeA-friendly 1/6/06
%


\documentclass{article}
\usepackage{a4,newlfont,dcolumn,copyrght,url,hevea}

\frenchspacing
\renewcommand{\copyrightyear}{1996}
\newcolumntype{d}[1]{D{#1}{#1}{-1}}
\newcommand{\coldash}{\multicolumn{1}{c|}{-}}
\setlength{\extrarowheight}{1pt}


\title{Tradeoffs in the implementation of the\\Beetle virtual machine}
\author{Reuben Thomas}
\date{20th June 1996}

\begin{document}
\maketitle


\section{Introduction}

Beetle is a virtual machine designed for running Forth compilers such as
pForth~\cite{beetledis}. It has been implemented in ANSI C~\cite{beetledis} for
portability, and, more recently, in ARM assembler. This paper describes the
tradeoffs within and between the two implementations, and compares them to
native Forth and C compilers.


\section{The virtual machine}

Beetle's computational model is a stack machine, and most of its instructions
are zero-operand, taking their operands implicitly from the data stack. The
only exceptions are branches, subroutine calls, and instructions to place
numeric literals on the stack. There are about ninety instructions, most of
which directly implement simple Forth words. A few others support specific
control and data constructs.

The instruction set is encoded as a byte stream grouped into words of four
bytes. Instructions are fetched a word at a time, and the bytes shifted out,
decoded and executed. This makes instruction fetch more efficient than for a
pure byte-stream encoding on many machines, though it introduces the overhead
of testing to see when a new instruction word must be fetched. Since the
instruction accumulator is arithmetically right-shifted after each
instruction has been decoded, this is achieved simply by making opcodes 00h
and FFh represent an instruction fetch.

Another consequence of grouping instructions into words is that branch
destinations are word-aligned. Thus, gaps sometimes appear in the
instruction stream where two control flows join; these are simply filled with
00h, so that when the gap is reached, execution proceeds immediately to the
next instruction word.

The only major problem caused by using a word-stream design is that endism,
or endianness, reared its ugly head. The solution adopted was to declare that
byte addressing is little-endian, but that it does not matter how the bytes
are actually stored. Thus, the only penalty is that byte addresses must have
their two least significant bits inverted on big-endian machines; there is no
penalty for word addressing, which is used for the majority of operations.

Address checking is optional in Beetle's design, because generally it is
impossible to implement so that there is no time penalty when it is turned
off, unless hardware checking is supported. Both implementations have
optional address checking that can be configured on or off at compile time.


\section{The implementations}

A byte-stream-encoded stack machine is possibly the easiest sort of virtual
machine to implement, and the change to a word stream only adds slight
complication to the instruction fetch. The obvious implementation method, and
that used in both implementations discussed here, is to switch on each
instruction opcode to branch to the action routine for the instruction.

\subsection{C Beetle}

The main aim of the C implementation of Beetle was that it should be easily
portable. To this end it was written in strict ANSI C, and implementation
dependencies were carefully isolated. In this it was successful (to date it
has been compiled on four different machine and operating system
configurations), but it sacrifices speed. The only concession to efficiency
was to have two versions of the interpretive loop, one for single stepping,
and the other for continuous running. On the other hand, it was quick to
write (the first version was completed in about three weeks, at a time when
the specification was still changing), easy to debug, and served as a useful
pattern for the other implementations.

The C implementation was provided with a simple command-line user interface,
including such facilities as single-stepping, disassembly and simple
profiling.

\subsection{ARM Beetle}

One of the design aims of Beetle was that it should be possible to produce
fast hand-coded versions for particular machines. Such a version was written
for the ARM processor (running under Acorn RISC OS). A branch table was used
for instruction dispatch, and the routine for decoding the next instruction
was expanded inline after each action routine. The registers of the virtual
machine, such as the program counter and stack pointers, were mapped on to
machine registers, and the action routines were carefully hand-coded. In
particular, instruction decode was coded in three instructions and
instruction fetch in one. Some optimisations, such as caching stack elements,
were expressly forbidden by Beetle's specification, so that different
implementations would be indistinguishable to programs running on them.

Address checking and single stepping were provided as independent optional
extras. The single stepping mode also counts how many times each instruction
is executed, to allow simple profiling.

The hand-coded Beetle is a direct replacement for the equivalent functions in
the C version.


\section{Measurements}

Benchmarks were run on both versions Beetle running in all operating modes
(that is, with and without address checking, and, in the case of the ARM
implementation, with without single-stepping and profiling). These consisted
of Forth programs compiled by the pForth compiler. For comparison, the
benchmarks were also run on a native-code version of pForth, and some of the
benchmarks were translated into C.

The main purpose of the benchmarks was to determine how closely interpretive
performance approached compiled performance; other issues considered were the
relative speeds of the virtual and real machines and the relative speed of
na\"{\i}vely-compiled Forth and optimised C (no optimising Forth compiler was
available; however it is safe to say that it is unlikely to have been faster
than the C compiler used, in any case).

All the benchmarks were run on an Acorn Risc PC 600, with a 30Mhz ARM610, in
single-tasking mode.

The benchmarks were timed, and the number of interpretive instructions
executed was counted. The static size of the benchmark and interpreter code
was also measured.

\subsection{The benchmarks}

Two main benchmarks were run, one I/O-bound, and one computation-intensive.
The first was a suite of tests designed to show that a Forth compiler
complies with the ANSI standard. It is essentially a long script which is
read from storage as the tests progress. The second benchmark was a pair of
implementations of Eratosthenes's sieve, counting, in this case, all the
primes up to 800,000, three times.

The code for the sieve benchmarks is given in appendix~\ref{code}; that for
the ANSI test suite is long and unenlightening.

The results of running the benchmarks are given in
tables~\ref{bench1tab}--~\ref{countstab}. All measurements were taken by hand
with a stopwatch, and averaged over several readings, which rarely differed
by more than 0.1s. All results are given to two significant figures.

When the number of instructions executed by Beetle were counted, only useful
instructions were included; the instruction fetch instruction was omitted
from the counts, and hence from the performance figures.

\begin{table}
\begin{center}
\begin{latexonly}
\begin{tabular}{|l|c|d{.}|} \hline
\rule[-2mm]{0mm}{6mm}\bf System & \bf Time/s & \multicolumn{1}{c|}{\bf mips}
\\ \hline
Native ARM                             & 4.3 & \coldash \\
Native Beetle (no checks)              & 5.6 & 1.1 \\
Native Beetle (address checks)         & 8.0 & 0.76 \\
Native Beetle (single stepping)        & 10 & 0.61 \\
Native Beetle (checking \& stepping)   & 14 & 0.44 \\
C Beetle (checking off)                & 16 & 0.38 \\
C Beetle (checking on)                 & 26 & 0.23 \\ \hline
\end{tabular}
\end{latexonly}
\begin{htmlonly}
\begin{tabular}{|l|c|l|} \hline
\rule[-2mm]{0mm}{6mm}\bf System & \bf Time/s & \bf mips
\\ \hline
Native ARM                             & 4.3 & \coldash \\
Native Beetle (no checks)              & 5.6 & 1.1 \\
Native Beetle (address checks)         & 8.0 & 0.76 \\
Native Beetle (single stepping)        & 10 & 0.61 \\
Native Beetle (checking \& stepping)   & 14 & 0.44 \\
C Beetle (checking off)                & 16 & 0.38 \\
C Beetle (checking on)                 & 26 & 0.23 \\ \hline
\end{tabular}
\end{htmlonly}
\caption{\label{bench1tab}Timings for the ANSI test suite benchmark}
\end{center}
\end{table}

\begin{table}
\begin{center}
\begin{latexonly}
\begin{tabular}{|l|c|d{.}|} \hline
\rule[-2mm]{0mm}{6mm}\bf System & \bf Time/s & \multicolumn{1}{c|}{\bf mips}
\\ \hline
Native ARM                             & 20 & \coldash \\
Native Beetle (no checks)              & 35 & 1.9 \\
Native Beetle (address checks)         & 54 & 1.3 \\
Native Beetle (single stepping)        & 70 & 0.97 \\
Native Beetle (checking \& stepping)   & 88 & 0.77 \\
C Beetle (checking off)                & 130 & 0.52 \\
C Beetle (checking on)                 & 200 & 0.34 \\
GNU C (optimised)                      & 2.0 & \coldash \\
GNU C (unoptimised)                    & 6.5 & \coldash \\ \hline
\end{tabular}
\end{latexonly}
\begin{htmlonly}
\begin{tabular}{|l|c|l|} \hline
\rule[-2mm]{0mm}{6mm}\bf System & \bf Time/s & \bf mips
\\ \hline
Native ARM                             & 20 & \coldash \\
Native Beetle (no checks)              & 35 & 1.9 \\
Native Beetle (address checks)         & 54 & 1.3 \\
Native Beetle (single stepping)        & 70 & 0.97 \\
Native Beetle (checking \& stepping)   & 88 & 0.77 \\
C Beetle (checking off)                & 130 & 0.52 \\
C Beetle (checking on)                 & 200 & 0.34 \\
GNU C (optimised)                      & 2.0 & \coldash \\
GNU C (unoptimised)                    & 6.5 & \coldash \\ \hline
\end{tabular}
\end{htmlonly}
\caption{\label{bench2tab}Timings for the primes benchmark (method 1)}
\end{center}
\end{table}

\begin{table}
\begin{center}
\begin{latexonly}
\begin{tabular}{|l|c|d{.}|} \hline
\rule[-2mm]{0mm}{6mm}\bf System & \bf Time/s & \multicolumn{1}{c|}{\bf mips}
\\ \hline
Native ARM                             & 14 & \coldash \\
Native Beetle (no checks)              & 20 & 1.8 \\
Native Beetle (address checks)         & 31 & 1.2 \\
Native Beetle (single stepping)        & 39 & 0.93 \\
Native Beetle (checking \& stepping)   & 50 & 0.73 \\
C Beetle (checking off)                & 75 & 0.49 \\
C Beetle (checking on)                 & 120 & 0.30 \\
GNU C (optimised)                      & 1.9 & \coldash \\
GNU C (unoptimised)                    & 6.4 & \coldash \\ \hline
\end{tabular}
\end{latexonly}
\begin{htmlonly}
\begin{tabular}{|l|c|l|} \hline
\rule[-2mm]{0mm}{6mm}\bf System & \bf Time/s & \bf mips
\\ \hline
Native ARM                             & 14 & \coldash \\
Native Beetle (no checks)              & 20 & 1.8 \\
Native Beetle (address checks)         & 31 & 1.2 \\
Native Beetle (single stepping)        & 39 & 0.93 \\
Native Beetle (checking \& stepping)   & 50 & 0.73 \\
C Beetle (checking off)                & 75 & 0.49 \\
C Beetle (checking on)                 & 120 & 0.30 \\
GNU C (optimised)                      & 1.9 & \coldash \\
GNU C (unoptimised)                    & 6.4 & \coldash \\ \hline
\end{tabular}
\end{htmlonly}
\caption{\label{bench3tab}Timings for the primes benchmark (method 2)}
\end{center}
\end{table}

\begin{table}
\begin{center}
\begin{latexonly}
\begin{tabular}{|l|d{.}|} \hline
\rule[-2mm]{0mm}{6mm}\bf Benchmark & \multicolumn{1}{c|}{\bf Instructions /
{\boldmath$ 10^6$}} \\ \hline
ANSI test suite & 6.1 \\
Primes method 1 & 67.8 \\
Primes method 2 & 36.4 \\ \hline
\end{tabular}
\end{latexonly}
\begin{htmlonly}
\begin{tabular}{|l|l|} \hline
\rule[-2mm]{0mm}{6mm}\bf Benchmark & \bf Instructions / {\boldmath$ 10^6$} \\ \hline
ANSI test suite & 6.1 \\
Primes method 1 & 67.8 \\
Primes method 2 & 36.4 \\ \hline
\end{tabular}
\end{htmlonly}
\caption{\label{countstab}Numbers of Beetle instructions executed in each
benchmark}
\end{center}
\end{table}

\subsection{Code size}

It is also interesting to compare static measurements of the different
implementations of Beetle. Table~\ref{sizetab} shows their sizes. The sizes
of the C versions are those of the object modules, and exclude the C runtime
system and other support code; the sizes of the native versions includes all
support code.

\begin{table}
\begin{center}
\begin{latexonly}
\begin{tabular}{|l|d{,}|} \hline
\rule[-2mm]{0mm}{6mm}\bf Implementation & \multicolumn{1}{c|}{\bf Size/bytes}
\\ \hline
Native Beetle (no checks) & 3,920\\
Native Beetle (address checks) & 6,692\\
Native Beetle (single stepping) & 8,328\\
Native Beetle (checking \& stepping) & 11,100\\
C Beetle (checking off) & 7,144 \\
C Beetle (checking on) & 13,864 \\ \hline
\end{tabular}
\end{latexonly}
\begin{htmlonly}
\begin{tabular}{|l|l|} \hline
\rule[-2mm]{0mm}{6mm}\bf Implementation & \bf Size/bytes
\\ \hline
Native Beetle (no checks) & 3,920\\
Native Beetle (address checks) & 6,692\\
Native Beetle (single stepping) & 8,328\\
Native Beetle (checking \& stepping) & 11,100\\
C Beetle (checking off) & 7,144 \\
C Beetle (checking on) & 13,864 \\ \hline
\end{tabular}
\end{htmlonly}
\caption{\label{sizetab}Sizes of different implementations of Beetle}
\end{center}
\end{table}

The benchmarks themselves were also measured. The ANSI test suite compiled
incrementally as it ran into 5,372 bytes, and the two primes tests compiled
into 356 bytes in total, with another 400,000 bytes used for the main array.


\section{Analysis}

The analysis is split into three parts; the first compares the two
implementations of the interpreter, the second compares the hand-coded
interpreter with the native pForth compiler, and the third makes comparisons
between all three, and with natively compiled and optimised C.

Many of the comparisons refer to table~\ref{speedtab}, which gives the mean
speed of each implementation relative to the fastest hand-coded version of
Beetle; the tests run in C are also included.

\begin{table}
\begin{center}
\begin{latexonly}
\begin{tabular}{|l|d{.}|} \hline
\rule[-2mm]{0mm}{6mm}\bf Implementation & \multicolumn{1}{c|}{\bf Relative
speed} \\ \hline
Native ARM                             & 1.5 \\
Native Beetle (no checks)              & 1.0 \\
Native Beetle (address checks)         & 0.66 \\
Native Beetle (single stepping)        & 0.52 \\
Native Beetle (checking \& stepping)   & 0.40 \\
C Beetle (checking off)                & 0.30 \\
C Beetle (checking on)                 & 0.19 \\
GNU C (optimised)                      & \multicolumn{1}{c|}{14} \\
GNU C (unoptimised)                    & 4.3 \\ \hline
\end{tabular}
\end{latexonly}
\begin{htmlonly}
\begin{tabular}{|l|l|} \hline
\rule[-2mm]{0mm}{6mm}\bf Implementation & \bf Relative speed \\ \hline
Native ARM                             & 1.5 \\
Native Beetle (no checks)              & 1.0 \\
Native Beetle (address checks)         & 0.66 \\
Native Beetle (single stepping)        & 0.52 \\
Native Beetle (checking \& stepping)   & 0.40 \\
C Beetle (checking off)                & 0.30 \\
C Beetle (checking on)                 & 0.19 \\
GNU C (optimised)                      & \multicolumn{1}{c|}{14} \\
GNU C (unoptimised)                    & 4.3 \\ \hline
\end{tabular}
\end{htmlonly}
\caption{\label{speedtab}Relative speeds}
\end{center}
\end{table}

\subsection{C Beetle \textit{versus} ARM Beetle}

There is no doubt that the hand-coded version of Beetle for the ARM processor
gave a marked speed improvement: the slowest native version is 1.3 times
quicker than the fastest C version; the fastest native Beetle is over five
times faster than C Beetle with address checks. The difference between
corresponding versions is a factor of about 3.4.

The only advantage that the native version of Beetle has over the ARM version
is that it is easily portable; it required between thirty minutes and a day's
work to port it to each of three different systems, while it took a week to
write and debug the ARM version. Though a week is far less time than it took
to write the C version originally, and not long for such performance gains,
the present author did have the benefit of a thorough knowledge of Beetle,
the ARM processor and the RISC OS environment; further hand-coded versions
would certainly take longer to produce.

The code size of an interpreter is important insofar as the interpreter will
run considerably faster if it can reside mainly in the processor's cache. The
ARM610 has a 4kb cache, so this goal is only achieved by the smallest version
of the native implementation; however, even within the interpreter there is
locality of code, so that even the larger C version will have a reasonable
cache hit rate. Typical high-performance microprocessors have much larger
caches, in any case.

According to these considerations, it seems that hand-coding Beetle was a
worthwhile effort on the machine in question; on higher performance machines
the extra effort is only worthwhile if the extra performance is required; it
is also not clear how much the performance would be improved on architectures
which lend themselves less readily to hand-coding than the delightful ARM.

\subsection{ARM Beetle \textit{versus} native pForth}

One of the most interesting results of the tests is that pForth runs only 1.5
times slower on the ARM version of Beetle than natively. This suggests that
there is little point producing native versions of pForth at all, and that a
good implementation of Beetle produces near-optimal execution of pForth.

An important point is that pForth is a na\"{\i}ve compiler, which performs no
optimisation. Since Beetle is designed specifically to execute compiled
Forth, a na\"{\i}ve compiler will produce good code for it; indeed, it is not
possible to perform much optimisation, as Forth does not have conventional
variables or scoping, and Beetle has no general purpose registers.

\subsection{General comparisons}

While the ARM implementation of Beetle compares well with a native Forth
compiler, the comparison with optimised C shows that there is still a long
way to go to obtain high performance. A factor of 14 is rather disheartening
in one sense, but in another it is irrelevant, as pForth is not designed to
produce fast code, but to provide a simple, comprehensible compiler, suitable
for teaching the principles of Forth compilation. Forth, too, does not aim to
give maximum speed, although Forth compilers usually provide assemblers so
that time-critical code can be hand coded; often, only a small portion of a
program needs to be coded to obtain C-like performance. Optimising Forth
compilers also exist.

It is also worth noting that the test system is a slow computer. Even the C
version of Beetle with address checking on runs acceptably fast on typical
large multi-user systems; indeed it runs at much the same sort of speed as
the hand-coded ARM Beetle. Native versions for such powerful systems would
run even faster, and provide more than adequate power for typical program
development.


\section{Conclusions}

The implementation of Beetle in ARM assembler demonstrates that resonable
performance can be obtained from interpreters. Interpreters are also quick to
implement, and easily portable, not much less so when hand-coded than when
implemented in a portable high-level language.

Code compiled by a na\"{\i}ve compiler can execute at near-optimal speed when
interpreted, provided that the interpreter is designed to support the
language being compiled. There seems to be little point implementing
simple-minded compilers in native code. It may even be a disadvantage for an
interpreted compiler to perform optimisation, as this will increase
compilation time, and in an interpretive environment, particularly one such
as Forth, which makes use of incremental compilation, this will interfere
with the rapid development cycle.

Even when implemented portably in a high-level language, interpreters can run
at a reasonable speed; although they cannot use machine-dependent tricks,
they can take advantage of optimising compilers to boost their speed.


\bibliographystyle{plain}
\bibliography{vm,rrt}

\newpage
\begin{appendix}
\section{Code for the sieve benchmarks}
\label{code}

These benchmarks are adapted from~\cite{forthsieve}.

\begin{verbatim}
400000 CONSTANT SIZE
VARIABLE FLAGS SIZE ALLOT

: DO-PRIME   ( method 1 )
   FLAGS SIZE 1 FILL
   0
   SIZE 0 DO
      FLAGS I + C@ IF
         I 2* 3 +  DUP I
         BEGIN  DUP SIZE < WHILE
            0  OVER FLAGS +  C!  OVER +
         REPEAT
         DROP DROP 1+
      THEN
   LOOP
   .   ." primes " ;
\end{verbatim}

\begin{verbatim}
: DO-PRIME-HI   ( method 2 )
   FLAGS SIZE 1 FILL
   0
   SIZE 0 DO
      I FLAGS + C@ IF
         I 2* 3 +  DUP I + SIZE < IF
            SIZE FLAGS +  OVER I +  FLAGS + DO
            0 I C!
            DUP +LOOP
         THEN
         DROP 1+
      THEN
   LOOP
   . ." primes " ;
\end{verbatim}
\end{appendix}


\end{document}
