%
% Documentation for Beetle User-Interface
%
% Reuben Thomas
%
% Started 27/2-8/5/95
%

\documentclass{article}
\usepackage[english]{babel}
\usepackage{a4,url}

% Alter some default parameters for general typesetting

\frenchspacing


% Macros

% Put an object in angle brackets for syntax descriptions

\newcommand{\angb}[1]{$\langle #1\rangle$}


\title{A simple user-interface for the\\Beetle virtual machine}
\author{Reuben Thomas}
\date{3rd June 2016}

\begin{document}
\maketitle


\section{Introduction}

The Beetle virtual machine \cite{beetle} provides a portable
environment for the pForth Forth compiler \cite{beetledis}, a compiler
for ANSI Standard Forth \cite{ANSIforth}. C Beetle \cite{cbeetle} is
a version of Beetle written in ISO C with POSIX APIs so that it is
itself easily portable. This paper describes a simple user-interface
for Beetle which uses C Beetle, as it is itself written in C, though
with suitable calling-interface glue it could be made to work with a
native code Beetle. The user-interface provides access to Beetle's
registers, allows the stacks to be displayed, and provides disassembly
and single-stepping.


\section{Compilation}

As supplied, the program consists of one C file, {\tt beetle.c}, which
is part of the C Beetle distribution, and is compiled when the entire
distribution is made, resulting in the executable file {\tt beetle}.
It can be made separately as the Make target {\tt beetle}. Run {\tt beetle --help} to see its syntax; most importantly, if given a single parameter, the name of an object file, the file is loaded and run.


\section{Initialisation}
\label{uifaceinit}

When the user-interface is started, an embedded Beetle is created, by default with $1,048,576$ cells of memory (4Mb); a different amount may be specified with the \verb|--memory| option. The registers which are set to system-dependent values are initialised as shown in table~\ref{uinittable}.

\begin{table}[htbp]
\begin{center}
\begin{tabular}{|c|c|} \hline
\rule[-2mm]{0mm}{6mm}\bf Register & \bf Initial value \\ \hline
{\tt EP} & 10h \\
{\tt MEMORY} & \emph{see above} \\
{\tt 'THROW} & 0h \\ \hline
\end{tabular}
\caption{\label{uinittable}Registers with system-dependent initial values}
\end{center}
\end{table}

{\tt I} is uninitialised. {\tt A} is set to zero: this has the effect that
when a {\tt STEP} or {\tt RUN} command is given, a {\tt NEXT} instruction
will be performed. Thus, when initialisation is performed by a {\tt LOAD}
command (see section~\ref{uifaceload}), the Beetle may be started with {\tt
RUN} or {\tt STEP} immediately after the {\tt LOAD} command, without the need
for {\tt FROM}. The memory from byte 16 upwards is zeroed.


\section{Commands}
\label{commands}

The user-interface is command-driven. All commands and register names may be
abbreviated to their first few letters; where commands start with the same
letters, there is a set order of precedence, aimed at giving the most
commonly used commands the shortest minimum abbreviations (see
section~\ref{abbrev}). All commands are case-insensitive.

If an unrecognised command is given, or the command has too few arguments, or
they are badly formed, an error message is displayed. Command lines
containing extraneous characters after a valid command are generally
accepted, and the extra characters ignored.

Numbers are all integral, and may be given in either decimal or hexadecimal
(which must be followed directly by ``h'' or ``H''), with an optional minus
sign.

For some arguments a machine instruction may also be used, preceded by ``{\tt
O}'', for opcode. The value of a machine instruction is its opcode. Opcodes
are byte-wide values; when used as a cell, the most significant three bytes
are set to zero.

The syntax of the commands is shown below; literal text such as command names
and other characters are shown in {\tt Typewriter font}; meta-parameters such
as numbers are shown in angle brackets, thus: \angb{number}. Vertical bars
separate alternatives, and square brackets enclose optional syntax.

There are three types of numeric meta-parameter: \angb{number}, which is any
number; \angb{address}, which is a valid address (between zero and {\tt
MEMORY} $-$ 1 inclusive, or zero and {\tt MEMORY} if assigning to {\tt RP},
{\tt R0}, {\tt SP} or {\tt S0}); and \angb{value}, which is a number or an
opcode.

\subsection{Registers}

Beetle's registers may be displayed by typing their name. The
registers may also (where appropriate) be assigned to using the syntax

\begin{center}\angb{register} $=$ \angb{value}\end{center}

where \angb{value} is in the form given in section~\ref{commands}. An error
message is displayed if an attempt is made to assign to a register such as
{\tt CHECKED}, which cannot be assigned to, or to assign an unaligned or out
of range address to a register which must hold an aligned address, such as
{\tt SP}. The {\tt FROM} command (see section~\ref{from}) should be used in
preference to assigning to {\tt EP}.

Two additional pseudo-registers are provided by the user-interface: they are
called {\tt S0} and {\tt R0}, and are the address of the base of the data and
return stacks respectively. They are set to the initial values of {\tt RP}
and {\tt SP}, and are provided so that they can be changed if the stacks are
moved, so the stack display commands will still work correctly.

The command {\tt REGISTERS} displays the contents of {\tt EP}, {\tt I} and
{\tt A}, useful when following the execution of a program.

\subsection{The stacks}

The stacks may be manipulated crudely using the registers {\tt SP} and {\tt
RP} but it is usually more convenient to use the commands

\begin{center}
{\tt >D} \angb{number}\\
{\tt D>}
\end{center}

which respectively push a number on to the data stack and pop one, displaying
it, and

\begin{center}
{\tt >R} \angb{number}\\
{\tt R>}
\end{center}

which do the same for the return stack.

The command {\tt DATA} displays the contents of the data stack, and {\tt
RETURN} the contents of the return stack. {\tt STACKS} displays both stacks.

If a stack underflows, or the base pointer or top of stack pointer is out of
range or unaligned, an appropriate error message is displayed.

\subsection{Memory}

The contents of an address may be displayed by giving the address as a
command. If the address is cell-aligned the whole cell is displayed,
otherwise the byte at that address is shown.

A larger section of memory may be displayed with the command {\tt DUMP},
which may be used in the two forms

\begin{center}
{\tt DUMP} \angb{address} \tt{+} \angb{number}\\
{\tt DUMP} \angb{address_1} \angb{address_2}
\end{center}

where the first displays \angb{number} bytes starting at address
\angb{address}, and the second displays memory from address \angb{address_1}
up to, but not including, address \angb{address_2}. An error message is
displayed if the start address is less than or equal to the end address or if
either address is out of range.

A command of the form

\begin{center}\angb{address} $=$ \angb{value}\end{center}

assigns the value \angb{value} to the address \angb{address}. If the address
is not cell-aligned, the value must fit in a byte, and only that byte is
assigned to. When assigning to an aligned memory location, a whole cell is
assigned unless the number given fits in a byte, and is given using the
minimum number of digits required. This should be noted the other way around:
to assign a byte-sized significand to a cell, it should be padded with a
leading zero.

\subsection{Execution}

The command {\tt INITIALISE} initialises Beetle as in section
\ref{uifaceinit}.

The command {\tt STEP} may be used to single-step through a program. It has
three forms:

\begin{center}
{\tt STEP}\\
{\tt STEP} \angb{number}\\
{\tt STEP TO} \angb{address}
\end{center}

Without arguments, {\tt STEP} executes one instruction. Given a number, {\tt
STEP} executes \angb{number} instructions. {\tt STEP TO} executes
instructions until {\tt EP} is equal to \angb{address}.

The command {\tt TRACE}, has the same syntax as {\tt STEP}, and performs the
same function; in addition, it performs the action of the {\tt REGISTERS}
command after each bForth instruction is executed.

The command {\tt RUN} allows Beetle to execute until it reaches a {\tt HALT}
instruction, if ever. The code passed to {\tt HALT} is then displayed. The
code is also displayed if a {\tt HALT} instruction is ever executed during a
{\tt STEP} command.

\label{from}
The command {\tt FROM} sets the point of execution. It has two forms:

\begin{center}
{\tt FROM}\\
{\tt FROM} \angb{address}
\end{center}

Without arguments, {\tt FROM} performs the function of Beetle's {\tt NEXT}
instruction, that is, it loads {\tt A} from the cell pointed to by {\tt EP},
and adds four to {\tt EP}. With an argument, {\tt FROM} sets {\tt EP} to
\angb{address}, and then performs the function of {\tt NEXT}. {\tt FROM}
should be used in preference to assigning directly to {\tt EP}.

The command {\tt DISASSEMBLE} disassembles bForth code. It may be used in the
two forms

\begin{center}
{\tt DISASSEMBLE} \angb{address} \tt{+} \angb{number}\\
{\tt DISASSEMBLE} \angb{address_1} \angb{address_2}
\end{center}

where the first disassembles \angb{number} bytes starting at address
\angb{address}, and the second from address \angb{address_1} up to, but not
including, address \angb{address_2}. The addresses must be cell-aligned, and
the number of bytes must be a multiple of four. An error message is displayed
if the start address is less than or equal to the end address, or if either
the address or number of bytes is not aligned or is out of range.

The command {\tt COUNTS} displays, if {\tt CHECKED} is 1, the number of times
that each Beetle instruction has been executed during {\tt STEP} or {\tt TRACE} execution since the last initialisation
(including loads).

\subsection{Object modules}

The command

\label{uifaceload}
\begin{center}{\tt LOAD} \angb{file} \angb{address}\end{center}

initialises Beetle as in section~\ref{uifaceinit}, then loads the object
module in file \angb{file} into memory at address \angb{address}. The address
must be cell-aligned; if it is not, or if the module would not fit in memory
at the address given, or there is some filing error, an error message is
displayed.

The command {\tt SAVE} saves an object module. It has the two forms

\begin{center}
{\tt SAVE} \angb{file} \angb{address} \tt{+} \angb{number}\\
{\tt SAVE} \angb{file} \angb{address_1} \angb{address_2}
\end{center}

where the first saves \angb{number} bytes starting at address \angb{address},
and the second saves from address \angb{address_1} up to, but not including,
address \angb{address_2}. The addresses must be cell-aligned, and the number
of bytes must be a multiple of four. An error message is displayed if the
start address is less than or equal to the end address, or if either the
address or number of bytes is not aligned or out of range.

The module is saved to the file \angb{file}. An error message is displayed if
there is some filing error, but no warning is given if a file of that name
already exists; it is overwritten.

\subsection{Exiting}

The command {\tt QUIT} exits the user-interface. No warning is given.


\section{Command abbreviations} \label{abbrev}

Below are listed the commands, each with its minimum abbreviation. Register
names may be abbreviated similarly, as long as the abbreviation does not
clash with a command abbreviation: if there is an ambiguity, it is assumed
that the command was intended.

\begin{table}[h]
\begin{center}
\begin{tabular}{|c|c|} \hline
\rule[-2mm]{0mm}{6mm}\bf Command & \bf Minimum abbreviation \\ \hline
{\tt >D} & {\tt >} \\
{\tt >R} & {\tt >R} \\
{\tt COUNTS} & {\tt C} \\
{\tt DISASSEMBLE} & {\tt D} \\
{\tt D>} & {\tt D>} \\
{\tt DATA} & {\tt DA} \\
{\tt DUMP} & {\tt DU} \\
{\tt FROM} & {\tt F} \\
{\tt INITIALISE} & {\tt I} \\
{\tt LOAD} & {\tt L} \\
{\tt QUIT} & {\tt Q} \\
{\tt REGISTERS} & {\tt R} \\
{\tt R>} & {\tt R>} \\
{\tt RETURN} & {\tt RET} \\
{\tt RUN} & {\tt RU} \\
{\tt STEP} & {\tt S} \\
{\tt SAVE} & {\tt SA} \\
{\tt STACKS} & {\tt ST} \\
{\tt TRACE} & {\tt T} \\ \hline
\end{tabular}
\caption{Minimum abbreviations of commands}
\end{center}
\end{table}


\bibliographystyle{plain}
\bibliography{vm,rrt}


\end{document}