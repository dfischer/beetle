%
% Documentation for C Beetle
%
% Reuben Thomas
%
% Started 1/12/94-8/5/95
%

\documentclass{article}
\usepackage[english]{babel}
\usepackage[utf8x]{inputenc}
\usepackage{a4,url,siunitx}


% Alter some default parameters for general typesetting

\frenchspacing


% Macros

% Lay out an external interface call

\newlength{\ifacewidth}\ifacewidth=\textwidth \advance\ifacewidth by -0.1in
\newlength{\innerwidth}\innerwidth=\ifacewidth \advance\innerwidth by -0.5in
\newcommand{\ifacec}[2]{\item[]\parbox{\ifacewidth}{\hspace*{2.5mm}{\tt #1}\\[0.5ex]\hspace*{0.4in}\parbox{\innerwidth}{#2}}}

% Font for stack pictures; macro \spic includes italic correction

\newfont{\spfont}{cmsltt10 scaled 1000}
\newcommand{\spic}[1]{{\spfont\setlength{\baselineskip}{\normalbaselineskip}#1\/}}


\title{An implementation of the Beetle virtual machine for POSIX}
\author{Reuben Thomas}
\date{7th February 2018}

\begin{document}
\maketitle


\section{Introduction}

The Beetle virtual machine~\cite{beetle} provides a portable environment
for the pForth Forth compiler~\cite{beetledis}, a compiler for ANSI Standard
Forth~\cite{ANSIforth}. To port pForth to a new CPU architecture or operating system, only Beetle need be rewritten. However, even this can be
avoided if Beetle is itself written in ISO C, since almost all systems have
an ISO C compiler available for them.

Writing Beetle in C necessarily leads to a loss of performance for a system
which is already relatively slow by virtue of using a virtual machine
rather than compiling native code. However, pForth is intended mainly as a
didactic tool, offering a concrete Forth environment which may be used to
explore the language, and particularly the implementation of the compiler, on
a simple architecture designed to support Forth. Thus speed is not crucial,
and on modern systems even a C implementation of Beetle can be expected to
run at an acceptable speed.

As well as the virtual machine, C Beetle provides a debugger, which is described in~\cite{beetleuiface}.

The Beetle virtual machine is described in~\cite{beetle}. This paper only
describes the features specific to this implementation.


\section{Omissions}

The {\tt OS} instruction is not implemented, as it depends on the operating
system of the host machine, and this implementation of Beetle is meant to be
portable. If executed, {\tt OS} does nothing.

The recursion instructions {\tt STEP} and {\tt RUN} are not implemented,
although they may be added in a future version.


\section{Using C Beetle}

This section describes how to compile C Beetle, and the exact manner in which
the interface calls and Beetle's memory and registers should be accessed.


\subsection{Configuration}
\label{configuration}

Beetle is written in ISO C99 using POSIX-1.2001 APIs.

The Beetle virtual machine is inherently 32-bit, but will run happily on systems with larger (or smaller) addresses.


\subsection{Compilation}

Beetle's build system is written with GNU autotools, and the user
needs only standard POSIX utilities to run it. Installation
instructions are provided in the top-level file {\tt README.md}.


\subsection{Registers and memory}

Beetle's registers are declared in {\tt beetle.h}. Their names correspond to
those given in~\cite[section~2.1]{beetle}, although some have been changed
to meet the rules for C identifiers. C Beetle does not allocate any
memory for Beetle, nor does it initialise any of the registers. C Beetle
provides the interface call {\bf init\_beetle()} to do this (see
section~\ref{usingcalls}).

The variables {\tt I}, {\tt A}, {\tt MEMORY}, {\tt BAD} and {\tt ADDRESS}
correspond exactly with the Beetle registers they represent, and may be read
and assigned to accordingly, bearing in mind the restrictions on their use
given in~\cite{beetle} (e.g. copies of {\tt BAD} and {\tt ADDRESS} must be
kept in Beetle's memory). {\tt THROW} is a pointer to the Beetle register
{\tt 'THROW}, so the expression {\tt *THROW} may be used as the Beetle
register. {\tt CHECKED} is the constant $1$; it may be read but not
assigned to.

{\tt EP}, {\tt M0}, {\tt SP} and {\tt RP} are machine pointers to the
locations in Beetle's address space to which the corresponding Beetle
registers point. Appropriate conversions (pointer addition or subtraction
with {\tt M0}) must therefore be made before using the value of one of these
variables as a Beetle address, and when assigning a Beetle address to one of
the corresponding registers. Examples of such conversions may be found in
{\tt step.c}, where the bForth instructions are implemented.

The main memory is accessed via {\tt M0}, which points to the first byte of
memory. Before Beetle is started by calling {\bf run()} or {\bf
single\_step()}, {\tt M0} must be set to point to a byte array which will be
Beetle's memory.

The main memory may be at most \SI{2}{GB}, and is always mapped into Beetle’s address space starting from address $0$. C Beetle provides the ability to map native memory blocks into the upper \SI{2}{GB} of the address space; see below.


\subsection{Extra library calls}

C Beetle provides the following extra {\tt LIB} calls.

\subsubsection{File system calls}

The file system calls correspond directly to ANS Forth words, as defined in~\cite{ANSIforth}.

\begin{center}
\begin{tabular}{|S[table-format=2.0]|l|} \hline
\rule[-2mm]{0mm}{6mm}\bf Number & \bf Forth word \\ \hline
4 & {\tt OPEN-FILE} \\
5 & {\tt CLOSE-FILE} \\
6 & {\tt READ-FILE} \\
7 & {\tt WRITE-FILE} \\
8 & {\tt FILE-POSITION} \\
9 & {\tt REPOSITION-FILE} \\
10 & {\tt FLUSH-FILE} \\
11 & {\tt RENAME-FILE} \\
12 & {\tt DELETE-FILE} \\ \hline
\end{tabular}
\end{center}

File access methods are bit-masks, composed as follows:

\begin{center}
\begin{tabular}{|c|c|} \hline
\rule[-2mm]{0mm}{6mm}\bf Bit value & \bf Meaning \\ \hline
1 & read \\
2 & write \\
4 & binary mode \\ \hline
\end{tabular}
\end{center}

To create a file, set both read and write bits to zero when calling {\tt OPEN-FILE}.

Two calls are provided to access command-line arguments passed to C Beetle (excluding any that it interprets itself). They are copied from Gforth.

\begin{center}
\begin{tabular}{|S[table-format=2.0]|l|l|l|} \hline
\rule[-2mm]{0mm}{6mm}\bf Number & \bf Forth word & \bf Stack effect & \bf Description \\ \hline
13 & {\tt ARGC} & \spic{-- u} & the number of arguments \\
14 & {\tt ARG} & \spic{u1 -- c-addr u2} & the \spic{u1}th argument \\ \hline
\end{tabular}
\end{center}


\subsection{Using the interface calls}
\label{usingcalls}

The operation of the specified interface calls is given in~\cite{beetle}. Here, the C prototypes corresponding to the idealised prototypes used in~\cite{beetle} are given.

Files to be loaded and saved are passed as C file descriptors. Thus, the
calling program must itself open and close the files.

\begin{description}
\ifacec{CELL run()}{The reason code returned by {\bf run()} is a Beetle
cell.}
\ifacec{CELL single\_step()}{The reason code returned by {\bf single\_step()}
is a Beetle cell.}
\ifacec{int load\_object(FILE *file, CELL *address)}{If a filing error
occurs, the return code is -3, which corresponds to a return value of {\tt
EOF} from {\bf getc()}.}
\ifacec{int save\_object(FILE *file, CELL *address, UCELL length)}{If a
filing error occurs, the return code is -3, which corresponds to a return
value of {\tt EOF} from {\bf putc()}.}
\end{description}

In addition to the required interface calls C Beetle provides an initialisation routine {\bf init\_beetle()} which, given a byte array, its size and an initial value for {\tt EP}, initialises Beetle:

\begin{description}
\ifacec{int init\_beetle(BYTE *b\_array, long size, UCELL e0)}{{\tt size} is
the length of {\tt b\_array} in {\em cells} (not bytes), and {\tt e0} is the
Beetle address to which EP will be set. The return value is -1 if {\tt e0} is
not aligned or out of range, and 0 otherwise. All the registers are
initialised as per~\cite{beetle}, and those held in Beetle's memory as well
are copied there.}
\end{description}

The following routines are provided to map system memory to Beetle’s address space:

\begin{description}
\ifacec{uint8\_t *himem\_addr(UCELL addr)}{Returns the native address corresponding to the given Beetle address (must be 80000000h or higher), if any. If the address is not mapped, {\tt NULL} is returned.}
\ifacec{UCELL himem\_allot(void *p, size\_t n)}{Map {\tt n} bytes pointed to by {\tt p} into Beetle’s address space, and return the address mapped. Addresses are mapped sequentially.}
\ifacec{UCELL himem\_align(void)}{Like Forth’s {\tt ALIGN}, rounds up the address at which the next high memory allocation will occur to the next cell.}
\end{description}

Programs which use C Beetle's interface must {\tt \#include} the header file
{\tt beetle.h} and be linked with the Beetle library. {\tt
opcodes.h}, which contains an enumeration type of Beetle's instruction set,
and {\tt debug.h}, which contains useful debugging functions such as
disassembly, may also be useful; they are not documented here.


\subsection{Other extras provided by C Beetle}

C Beetle provides the following extra quantities and macro in {\tt beetle.h}
which are useful for programming with Beetle:

\begin{description}
\item[{\tt B\_TRUE}:] a cell with all bits set, which Beetle uses as a true
flag.
\item[{\tt B\_FALSE}:] a cell with all bits clear, which Beetle uses as a
false flag.
\item[{\tt CELL\_W}:] the width of a cell in bytes (4).
\item[{\tt POINTER\_W}:] the width of a machine pointer in cells.
\item[{\tt NEXT}:] a macro which performs the action of the {\tt NEXT}
instruction.
\item[{\tt CELL\_pointer}:] a union with members {\tt CELL cells[POINTER\_W]} and {\tt void (*pointer)(void)}, which allow a function pointer suitable for the {\tt LINK} instruction to be easily stored and retrieved. It is assumed that the pointer is pushed on to the stack starting with {\tt cells[0]} and ending with {\tt cells[POINTER\_W~$-$~1]}.
\end{description}


\bibliographystyle{plain}
\bibliography{vm,rrt}


\end{document}
